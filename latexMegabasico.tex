% Este documento esta pensado para que lo veas ya compilado, el pdf
% Linda explicacion sobre latex: https://help.ubuntu.com/community/LaTeX
% Es un markup language (como html) y cumple el mismo proposito. Se puede convertir a pdf y html
% Aca hay una referencia del lenguaje http://www.ntg.nl/doc/biemesderfer/ltxcrib.pdf
\begin{Document}

\section{Tags mas usados...(se explican solos) de Layout}
\verb 
/
\section{Primera Seccion}
  \subsection{Primera Subseccion}
    \begin{itemize}
    \Item
    Soy el primer Item
    \Item
    Soy el segundo item
    \end{itemize}
    \subsubsection{Primera Subsubseccion}
      \subsubsection*{Hijo de subsubseccion 1}
      Estoy dentro del primer subtitulo de la Primera Subsubseccion
      \subsubsection*{Hijo de subsubseccion 2}
      Estoy dentro del segundo subtitulo de la Primera Subsubseccion
    \subsubsection{Segunda Subsubseccion}
  \subsection{Segunda Subseccion}
\section{Segunda Seccion}
/

\section{Avanzado}
  \subsection{misc}
  \begin{itemize}
  \Item
  Indice: es en realidad \\tableofcontents y se genera a partir de las secciones, capitulos,etc. No va andar si corriste una sola vez por que donde se genera tdv no se conocen las secciones.
  \Item 
  Referencias: \\label{marker} / \\ref{marker} . Idem indice(tableofcontents) se generan con la segunda corrida
  \item
  Apendice e inclusiones de archivos pdf:
  \end{itemize}

  \subsection{math}
  \begin{itemize}
  \Item
  Con  ( signo pesos) una vez entro en modo math de forma inline (o sea mete una formulita en un p\'arrafo)
  \Item 
  Con (signo pesos signo pesos) entro en modo math pero en forma block (o sea formula en una linea separada)
  \end{itemize}

  \subsection{otros tags}
  \begin{itemize}
  \Item
  Con  ( signo pesos) una vez entro en modo math de forma inline (o sea mete una formulita en un p\'arrafo)
  \Item 
  Con (signo pesos signo pesos) entro en modo math pero en forma block (o sea formula en una linea separada)
  \end{itemize}
  

\end{Document}

% TO DO:
% QUE HACE VERB QUE USE ACA ? USALO EN LO DE TABLEOFCONTENTS Y SIGNO PESOS
% DOCUMENTO MINIMO CON EL TEMA DE LOS IDIOMAS Y LOS ACENTOS

